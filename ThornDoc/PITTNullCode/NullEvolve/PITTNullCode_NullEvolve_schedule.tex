
\section{Schedule} 


\parskip = 0pt


\noindent This section lists all the variables which are assigned storage by thorn PITTNullCode/NullEvolve.  Storage can either last for the duration of the run ({\bf Always} means that if this thorn is activated storage will be assigned, {\bf Conditional} means that if this thorn is activated storage will be assigned for the duration of the run if some condition is met), or can be turned on for the duration of a schedule function.


\subsection*{Storage}

\hspace{5mm}

 \begin{tabular*}{160mm}{ll} 

{\bf Always:}& {\bf Conditional:} \\ 
 NullVars::realcharfuncs[2] &  NullVars::cmplxcharfuncs\_aux[2]\\ 
 NullVars::cmplxcharfuncs\_basic[2] & ~\\ 
 NullVars::null\_mask & ~\\ 
 NullGrid::EG\_mask NullGrid::EQ\_mask NullGrid::EV\_mask & ~\\ 
 eth4\_mask dissip\_mask & ~\\ 
 diagtmp aux\_mask & ~\\ 
 Jrad dxJrad & ~\\ 
~ & ~\\ 
\end{tabular*} 


\subsection*{Scheduled Functions}
\vspace{5mm}

\noindent {\bf CCTK\_POSTINITIAL}   (conditional) 

\hspace{5mm} nullevol\_initial 

\hspace{5mm}{\it null init data } 


\hspace{5mm}

 \begin{tabular*}{160mm}{cll} 
~ & After:  & harmidata\_init\_to\_adm \\ 
~& ~ &harmidata\_pinit\_to\_adm\\ 
~& ~ &mol\_fillalllevels\\ 
~& ~ &adm\_bssn\_init\\ 
~ & Options:  & global \\ 
~ & Type:  & group \\ 
\end{tabular*} 


\vspace{5mm}

\noindent {\bf CCTK\_INITIAL}   (conditional) 

\hspace{5mm} nullevol\_initializearrays 

\hspace{5mm}{\it initialize all arrays to large values } 


\hspace{5mm}

 \begin{tabular*}{160mm}{cll} 
~ & Before:  & nullevol\_initial \\ 
~ & Language:  & fortran \\ 
~ & Type:  & function \\ 
\end{tabular*} 


\vspace{5mm}

\noindent {\bf NullEvol\_BoundaryInit}   (conditional) 

\hspace{5mm} nullevol\_bdry\_flat 

\hspace{5mm}{\it give flat boundary conditions for the null metric } 


\hspace{5mm}

 \begin{tabular*}{160mm}{cll} 
~ & Language:  & fortran \\ 
~ & Options:  & global \\ 
~ & Type:  & function \\ 
\end{tabular*} 


\vspace{5mm}

\noindent {\bf NullEvol\_BoundaryInit}   (conditional) 

\hspace{5mm} nullevol\_bdry\_whitehole 

\hspace{5mm}{\it give white hole boundary conditions for the null metric } 


\hspace{5mm}

 \begin{tabular*}{160mm}{cll} 
~ & Language:  & fortran \\ 
~ & Options:  & global \\ 
~ & Type:  & function \\ 
\end{tabular*} 


\vspace{5mm}

\noindent {\bf NullEvol\_BoundaryInit}   (conditional) 

\hspace{5mm} nullevol\_bdry\_randomj 

\hspace{5mm}{\it give random j boundary conditions for the null metric } 


\hspace{5mm}

 \begin{tabular*}{160mm}{cll} 
~ & Language:  & fortran \\ 
~ & Options:  & global \\ 
~ & Type:  & function \\ 
\end{tabular*} 


\vspace{5mm}

\noindent {\bf NullEvol\_Boundary}   (conditional) 

\hspace{5mm} nullevol\_bdry\_flat 

\hspace{5mm}{\it give flat boundary conditions for the null metric } 


\hspace{5mm}

 \begin{tabular*}{160mm}{cll} 
~ & Language:  & fortran \\ 
~ & Options:  & global \\ 
~ & Type:  & function \\ 
\end{tabular*} 


\vspace{5mm}

\noindent {\bf NullEvol\_Boundary}   (conditional) 

\hspace{5mm} nullevol\_bdry\_whitehole 

\hspace{5mm}{\it give white hole boundary conditions for the null metric } 


\hspace{5mm}

 \begin{tabular*}{160mm}{cll} 
~ & Language:  & fortran \\ 
~ & Options:  & global \\ 
~ & Type:  & function \\ 
\end{tabular*} 


\vspace{5mm}

\noindent {\bf NullEvol\_Boundary}   (conditional) 

\hspace{5mm} nullevol\_bdry\_randomj 

\hspace{5mm}{\it give random j boundary conditions for the null metric } 


\hspace{5mm}

 \begin{tabular*}{160mm}{cll} 
~ & Language:  & fortran \\ 
~ & Options:  & global \\ 
~ & Type:  & function \\ 
\end{tabular*} 


\vspace{5mm}

\noindent {\bf CCTK\_EVOL}   (conditional) 

\hspace{5mm} nullevol\_resettop 

\hspace{5mm}{\it reset top values } 


\hspace{5mm}

 \begin{tabular*}{160mm}{cll} 
~ & Before:  & nullevol\_boundary \\ 
~ & Language:  & fortran \\ 
~ & Options:  & global \\ 
~ & Type:  & function \\ 
\end{tabular*} 


\vspace{5mm}

\noindent {\bf CCTK\_EVOL}   (conditional) 

\hspace{5mm} nullevol\_boundary 

\hspace{5mm}{\it boundary data } 


\hspace{5mm}

 \begin{tabular*}{160mm}{cll} 
~ & After:  & harmevol\_to\_adm \\ 
~& ~ &mol\_evolution\\ 
~ & Options:  & global \\ 
~ & Type:  & group \\ 
\end{tabular*} 


\vspace{5mm}

\noindent {\bf CCTK\_EVOL}   (conditional) 

\hspace{5mm} nullevol\_step 

\hspace{5mm}{\it evolution } 


\hspace{5mm}

 \begin{tabular*}{160mm}{cll} 
~ & After:  & nullevol\_boundary \\ 
~ & Language:  & fortran \\ 
~ & Options:  & global \\ 
~ & Storage:  & distmp \\ 
~ & Type:  & function \\ 
\end{tabular*} 


\vspace{5mm}

\noindent {\bf NullEvol\_Initial}   (conditional) 

\hspace{5mm} nullevol\_initialdata 

\hspace{5mm}{\it give j on the initial null hypersurface } 


\hspace{5mm}

 \begin{tabular*}{160mm}{cll} 
~ & After:  & nullevol\_boundaryinit \\ 
~ & Language:  & fortran \\ 
~ & Options:  & global \\ 
~ & Type:  & function \\ 
\end{tabular*} 


\vspace{5mm}

\noindent {\bf NullEvol\_Initial}   (conditional) 

\hspace{5mm} nullevol\_boundaryinit 

\hspace{5mm}{\it boundary data for the characteristic data } 


\hspace{5mm}

 \begin{tabular*}{160mm}{cll} 
~ & Before:  & nullevol\_initialdata \\ 
~& ~ &nullevol\_initialslice\\ 
~ & Type:  & group \\ 
\end{tabular*} 


\vspace{5mm}

\noindent {\bf NullEvol\_Initial}   (conditional) 

\hspace{5mm} nullevol\_initialslice 

\hspace{5mm}{\it construct null metric on the initial null hypersurface } 


\hspace{5mm}

 \begin{tabular*}{160mm}{cll} 
~ & After:  & nullevol\_boundaryinit \\ 
~& ~ &nullevol\_initialdata\\ 
~ & Language:  & fortran \\ 
~ & Options:  & global \\ 
~ & Type:  & function \\ 
\end{tabular*} 


\vspace{5mm}

\noindent {\bf NullEvol\_Initial}   (conditional) 

\hspace{5mm} nullevol\_diag 

\hspace{5mm}{\it diagnostics of the characteristic code } 


\hspace{5mm}

 \begin{tabular*}{160mm}{cll} 
~ & After:  & nullevol\_initialslice \\ 
~ & Language:  & fortran \\ 
~ & Options:  & global \\ 
~ & Storage:  & diagtmp \\ 
~ & Type:  & function \\ 
\end{tabular*} 


\vspace{5mm}

\noindent {\bf CCTK\_EVOL}   (conditional) 

\hspace{5mm} nullevol\_diag 

\hspace{5mm}{\it diagnostics of the characteristic code } 


\hspace{5mm}

 \begin{tabular*}{160mm}{cll} 
~ & After:  & nullevol\_step \\ 
~ & Language:  & fortran \\ 
~ & Options:  & global \\ 
~ & Storage:  & diagtmp \\ 
~ & Type:  & function \\ 
\end{tabular*} 



\vspace{5mm}\parskip = 10pt 
