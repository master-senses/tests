\documentclass{article}

% Use the Cactus ThornGuide style file
% (Automatically used from Cactus distribution, if you have a 
%  thorn without the Cactus Flesh download this from the Cactus
%  homepage at www.cactuscode.org)
\usepackage{../../../../../doc/latex/cactus}

\newlength{\tableWidth} \newlength{\maxVarWidth} \newlength{\paraWidth} \newlength{\descWidth} \begin{document}

\title{FleshInfo}
\author{Gabrielle Allen}
\date{$ $Date$ $}

\maketitle

% Do not delete next line
% START CACTUS THORNGUIDE

\begin{abstract}
Demonstrates obtaining information about computational infrastructure
from the flesh
\end{abstract}

\section{Purpose}

This thorn demonstrates using flesh interfaces to obtain information
about the computational infrastructure available to a simulation. 
In this case, the information obtained is simply reported to standard
output. This thorn doesn't exhaust all the available information, although
hopefully it will be expanded to show all the applicable APIs in practise.
Notable exceptions at the moment include information about parameters, 
scheduled functions and grid variables.

\section{Reduction Operators}

FleshInfo provides a list or registered reduction operators, along 
with the thorn and implementation which registered them.

\section{Interpolation Operators}

FleshInfo provides a list or registered interpolation operators, along 
with the thorn and implementation which registered them.

\section{IO Methods}

FleshInfo provides a list or registered IO methods, along 
with the thorn and implementation which registered them.

\section{Coordinates}
FleshInfo reports on
\begin{itemize}
\item Registered coordinate systems, their dimensions and the thorn
	and implementation which registered them
\item For each coordinate system, a list of coordinates, their
	directions, the computational range and the physical index range.
\end{itemize}

% Do not delete next line
% END CACTUS THORNGUIDE



\section{Parameters} 


\parskip = 0pt
\parskip = 10pt 

\section{Interfaces} 


\parskip = 0pt

\vspace{3mm} \subsection*{General}

\noindent {\bf Implements}: 

fleshinfo
\vspace{2mm}

\vspace{5mm}\parskip = 10pt 

\section{Schedule} 


\parskip = 0pt


\noindent This section lists all the variables which are assigned storage by thorn CactusExamples/FleshInfo.  Storage can either last for the duration of the run ({\bf Always} means that if this thorn is activated storage will be assigned, {\bf Conditional} means that if this thorn is activated storage will be assigned for the duration of the run if some condition is met), or can be turned on for the duration of a schedule function.


\subsection*{Storage}NONE
\subsection*{Scheduled Functions}
\vspace{5mm}

\noindent {\bf CCTK\_INITIAL} 

\hspace{5mm} interpinfo 

\hspace{5mm}{\it information about interpolation operators } 


\hspace{5mm}

 \begin{tabular*}{160mm}{cll} 
~ & Language:  & c \\ 
~ & Type:  & function \\ 
\end{tabular*} 


\vspace{5mm}

\noindent {\bf CCTK\_INITIAL} 

\hspace{5mm} reduceinfo 

\hspace{5mm}{\it information about reduction operators } 


\hspace{5mm}

 \begin{tabular*}{160mm}{cll} 
~ & Language:  & c \\ 
~ & Type:  & function \\ 
\end{tabular*} 


\vspace{5mm}

\noindent {\bf CCTK\_INITIAL} 

\hspace{5mm} ioinfo 

\hspace{5mm}{\it information about io methods } 


\hspace{5mm}

 \begin{tabular*}{160mm}{cll} 
~ & Language:  & c \\ 
~ & Type:  & function \\ 
\end{tabular*} 


\vspace{5mm}

\noindent {\bf CCTK\_INITIAL} 

\hspace{5mm} coordinfo 

\hspace{5mm}{\it information about coordinate systems } 


\hspace{5mm}

 \begin{tabular*}{160mm}{cll} 
~ & Language:  & c \\ 
~ & Type:  & function \\ 
\end{tabular*} 



\vspace{5mm}\parskip = 10pt 
\end{document}
