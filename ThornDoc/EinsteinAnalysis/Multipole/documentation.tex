\documentclass{article}
% Use the Cactus ThornGuide style file
% (Automatically used from Cactus distribution, if you have a
%  thorn without the Cactus Flesh download this from the Cactus
%  homepage at www.cactuscode.org)
\usepackage{../../../../../doc/latex/cactus}

\newlength{\tableWidth} \newlength{\maxVarWidth} \newlength{\paraWidth} \newlength{\descWidth} \begin{document}

% The author of the documentation
\author{Ian Hinder}

% The title of the document (not necessarily the name of the Thorn)
\title{Multipole}

% the date your document was last changed, if your document is in CVS,
% please use:
%    \date{$ $Date: 2004-01-07 14:12:39 -0600 (Wed, 07 Jan 2004) $ $}
\date{01 Jun 2010}

\maketitle

% Do not delete next line
% START CACTUS THORNGUIDE

% Add all definitions used in this documentation here
%   \def\mydef etc

% Add an abstract for this thorn's documentation
\begin{abstract}
The Multipole thorn performs spherical harmonic mode decomposition
of Cactus grid functions on coordinate spheres.
\end{abstract}

% The following sections are suggestive only.
% Remove them or add your own.

\section{Introduction}
This thorn allows the user to compute the coefficients of the
spherical harmonic expansion of a field stored in a Cactus grid
function on coordinate spheres of given radii.  A set of radii for
these spheres, as well as the number of angular points to use, can be
specified.  Complex fields can be used, but they must be stored in
pairs of real Cactus grid functions (the \verb|CCTK_COMPLEX| type is
not supported).

\section{Physical System}
The angular dependence of a field $u(t, r, \theta, \varphi)$ can be
expanded in spin-weight $s$ spherical harmonics
\cite{Goldberg:1966uu}:

\begin{eqnarray}
  u(t, r, \theta, \varphi) = \sum_{l=0}^\infty \sum_{m=-l}^l C^{lm}(t,r) {}_s Y_{lm}(\theta,\varphi)
\end{eqnarray}

where the coefficients $C^{lm}(t,r)$ are given by

\begin{eqnarray}
C^{lm}(t, r) = \int {}_s Y_{lm}^* u(t, r, \theta, \varphi) r^2 d \Omega \label{eqn:clmint}
\end{eqnarray}

At any given time, $t$, this thorn can compute $C^{lm}(t,r)$ for a
number of grid functions on several coordinate spheres with radii $r_i$.
The coordinate system of the Cactus grids must be Cartesian and the
coordinates $r$, $\theta$, $\varphi$ are related to $x$, $y$ and $z$
by the usual transformation between Cartesian and spherical polar
coordinates ($\theta$ is the polar angle and $\varphi$ is the
azimuthal angle).

The spin-weighted spherical harmonics are computed using Eq.~3.1 in
Ref.~\cite{Goldberg:1966uu}.

\section{Numerical Implementation}

The coordinate sphere on which the multipolar decomposition is
performed is represented internally as a 2-dimensional grid evenly
spaced in $\theta$ and $\varphi$ with coordinates

\begin{eqnarray}
\theta_k &=& \frac{\pi k}{n_\theta} \quad k = 0, 1, ..., n_\theta \\
\varphi_k &=& \frac{2\pi k}{n_\varphi} \quad k = 0, 1, ..., n_\varphi,
\end{eqnarray}

so $n_\theta$ and $n_\varphi$ count the number of cells (not the
number of points).  The gridfunction to be decomposed, $u$, is first
interpolated from the 3D Cactus grid onto this 2D grid at a given
radius, $r_i$, and the $C^{lm}$ are computed by evaluating the integral
in Eq.~\ref{eqn:clmint} for different values of $l$ and $m$.  The
interpolation is performed using the Cactus interpolation interface,
so any Cactus interpolator can be used.  The numerical method used for
interpolation should be specified in the documentation for the thorn
which provides it.  One such thorn is AEILocalInterp.  The integration
is performed using either the midpoint rule, yielding a result which
is second order accurate in the angular spacings $\Delta \theta =
\pi/n_\theta$ and $\Delta \varphi = 2\pi/n_\theta$, or Simpson's rule,
which is fourth order accurate.

\section{Using This Thorn}

\subsection{Obtaining This Thorn}

This thorn is available as part of the Einstein Toolkit via:

{\tt svn checkout https://svn.einsteintoolkit.org/cactus/EinsteinAnalysis/Multipole/trunk Multipole}

or by using the Einstein Toolkit \verb|GetComponents| script and thornguide.

\subsection{Basic Usage}

Suppose that you have a real grid function, $u$, for which you want to
compute the spherical harmonic coefficients $C^{lm}$.  Start by
including Multipole and an interpolator (for example AEILocalInterp)
in the ActiveThorns line of your parameter file (for interpolation
thorns other than AEILocalInterp, you will need to modify the
\verb|interpolator| and \verb|interpolator_options| accoording to the
documentation of the interpolation thorn).  Next decide the number and
radii of the coordinate spheres on which you want to decompose.  Set
the number of spheres with the \verb|nradii| parameter, and the radii
themselves with the \verb|radius[i]| parameters (the indices $i$ are
zero-based).  For example,

\begin{verbatim}
ActiveThorns = ".... AEILocalInterp Multipole"

Multipole::nradii       = 3
Multipole::radius[0]    = 10
Multipole::radius[1]    = 20
Multipole::radius[2]    = 30
Multipole::variables    = "MyThorn::u"
\end{verbatim}

The default parameters will compute all $l = 2$ modes assuming a
spin-weight $s = 0$ on every iteration of your simulation.  The
coefficients $C^{lm}$ will be output in the files with names of the
form \verb|mp_<var>_l<lmode>_m<mmode>_r<rad>.asc|, for example

\begin{verbatim}
mp_u_l2_m2_r10.00.asc
mp_u_l2_m-1_r20.00.asc
\end{verbatim}

For the filename, the radius is converted to a string with two decimal
places, which should be sufficient.  These are ASCII files where each
line has columns

$t$, $\mathrm{Re} \, C^{lm}$, $\mathrm{Im} \, C^{lm}$.

\subsection{Special Behaviour}
Often it will be necessary to go beyond the basic usage described in the
previous section.

\subsubsection{Higher modes}
By default, Multipole computes only $l = 2$ modes.  You can choose
whether to extract a single mode, or all modes from $l = $
\verb|l_min| to $l = $ \verb|l_mode|, with $|m| \le $ \verb|m_mode|.
This is controlled by the \verb|mode_type| parameter, which can be set
to \verb|"all_modes"| or \verb|"specific_mode"|.  The parameters
\verb|l_min| and \verb|l_mode| specify the lowest and highest modes to
compute.  The parameter \verb|m_mode| specifies up to which value of
$m$ to compute.  When using \verb|mode_type| = \verb|"specific_mode"|,
the mode $l$ and $m$ are given by \verb|l_mode| and \verb|m_mode|
respectively.

\subsubsection{Variable options}
Several variable-specific options can be listed as {\em tags} in the
\verb|variables| parameter:
\begin{verbatim}
Multipole::variables = "<imp>::<var>{<tagname> = <tagvalue> ... } ..."
\end{verbatim}
Valid tags are:

\begin{tabular}{lp{0.9\columnwidth}}
  cmplx & A string giving the fully qualified variable name for the imaginary part of a complex field, assuming that \verb|<imp>::<var>| is the real part. \\
  name & A string giving an alias to use for the decomposed variable in the output filename, for use in the case of complex variable when otherwise the name of the real part would be used, which might be confusing. \\
  sw & The integer spin-weight of the spherical harmonic decomposition to use.
\end{tabular}

Strings should be enclosed in single quotes within the list of tags.

\subsubsection{Complex variables}
In order to decompose a complex quantity, Multipole currently requires
that the field is stored in two separate \verb|CCTK_REAL| grid
functions, one for the real and one for the imaginary part.  Suppose
the complex function $u$ is stored in two gridfunctions \verb|u_re|
and \verb|u_im|.  In order to correctly decompose $u$, specify the
real variable in the variables parameter, and use the tag \verb|cmplx|
to specify the name of its imaginary companion:
\begin{verbatim}
Multipole::variables = "MyThorn::u_re{cmplx = 'MyThorn::u_im'}"
\end{verbatim}

\subsubsection{Renaming variables}
In some cases, you might want the name of the variable in the output
filename to be different to the name of the grid function.  This can
be done by setting the \verb|name| tag of the variable:
\begin{verbatim}
Multipole::variables = "MyThorn::u{name = 'myfunction'}"
\end{verbatim}
For example, in the case of a complex variable where the output file
contains the name of the real part, you can rename it as follows:
\begin{verbatim}
Multipole::variables = "MyThorn::u_re{cmplx = 'MyThorn::u_im' name = 'u'}"
\end{verbatim}

\subsubsection{Spin weight}
Depending on the nature of the field to be decomposed, a 
spin-weight other than 0 might be required in the spherical harmonic basis.  Use
the tag \verb|sw| for this
\begin{verbatim}
Multipole::variables = "MyThorn::u_re{cmplx = 'MyThorn::u_im' name = 'u' sw = -2}"
\end{verbatim}

\subsubsection{Interpolator options}
The interpolator to be used can be specified in the
\verb|interpolator_name| parameter, and a string containing
interpolator parameters can be specified in the
\verb|interpolator_pars| parameter.  See the interpolator (for example
AEIThorns/AEILocalInterp) documentation for details of interpolators
available and their options.

\subsubsection{Output}
When used with mesh-refinement, it is common to require mode
decomposition less frequently than every iteration.  The parameter
\verb|out_every| can be used to control this.  1D and 2D output of the
coordinate spheres can be enabled using \verb|out_1d_every| and
\verb|out_2d_every|.

\subsection{Interaction With Other Thorns}

This thorn obtains grid function data via the standard Cactus
interpolator interface.  To use this, one needs the parallel driver
(for example PUGHInterp or CarpetInterp) as well as the low-level
interpolator (e.g. AEILocalInterp).

\subsection{Examples}

To use this thorn with WeylScal4 to compute modes of the complex
$\Psi_4$ variable, one could use the following example:

\begin{verbatim}
ActiveThorns = ".... WeylScal4 CarpetInterp AEILocalInterp Multipole"

Multipole::nradii    = 3
Multipole::radius[0] = 10
Multipole::radius[1] = 20
Multipole::radius[2] = 30
Multipole::variables = "WeylScal4::Psi4r{sw=-2 cmplx='WeylScal4::Psi4i'}"
Multipole::l_mode    = 4
Multipole::m_mode    = 4
\end{verbatim}

\section{History}

This thorn was developed in the Penn State Numerical Relativity group
and contributed to the Einstein Toolkit.

\subsection{Acknowledgements}

This thorn was written by Ian Hinder and Andrew Knapp, with
contributions from Eloisa Bentivegna and Shaun Wood.

%\bibliography{multipole}
%\bibliographystyle{plain}

\begin{thebibliography}{1}
\bibitem{Goldberg:1966uu}
J.~N. Goldberg, A.~J. MacFarlane, E.~T. Newman, F.~Rohrlich, and E.~C.~G.
  Sudarshan.
\newblock {Spin s spherical harmonics and edth}.
\newblock {\em J. Math. Phys.}, 8:2155, 1967.
\end{thebibliography}

% Do not delete next line
% END CACTUS THORNGUIDE



\section{Parameters} 


\parskip = 0pt

\setlength{\tableWidth}{160mm}

\setlength{\paraWidth}{\tableWidth}
\setlength{\descWidth}{\tableWidth}
\settowidth{\maxVarWidth}{test\_mode\_proportional\_to\_r}

\addtolength{\paraWidth}{-\maxVarWidth}
\addtolength{\paraWidth}{-\columnsep}
\addtolength{\paraWidth}{-\columnsep}
\addtolength{\paraWidth}{-\columnsep}

\addtolength{\descWidth}{-\columnsep}
\addtolength{\descWidth}{-\columnsep}
\addtolength{\descWidth}{-\columnsep}
\noindent \begin{tabular*}{\tableWidth}{|c|l@{\extracolsep{\fill}}r|}
\hline
\multicolumn{1}{|p{\maxVarWidth}}{coord\_system} & {\bf Scope:} restricted & STRING \\\hline
\multicolumn{3}{|p{\descWidth}|}{{\bf Description:}   {\em What is the coord system?}} \\
\hline{\bf Range} & &  {\bf Default:} cart3d \\\multicolumn{1}{|p{\maxVarWidth}|}{\centering .*} & \multicolumn{2}{p{\paraWidth}|}{Any smart string will do} \\\hline
\end{tabular*}

\vspace{0.5cm}\noindent \begin{tabular*}{\tableWidth}{|c|l@{\extracolsep{\fill}}r|}
\hline
\multicolumn{1}{|p{\maxVarWidth}}{enable\_test} & {\bf Scope:} restricted & BOOLEAN \\\hline
\multicolumn{3}{|p{\descWidth}|}{{\bf Description:}   {\em whether to set a spherical harmonic in the 'harmonic' grid functions}} \\
\hline & & {\bf Default:} no \\\hline
\end{tabular*}

\vspace{0.5cm}\noindent \begin{tabular*}{\tableWidth}{|c|l@{\extracolsep{\fill}}r|}
\hline
\multicolumn{1}{|p{\maxVarWidth}}{hdf5\_chunk\_size} & {\bf Scope:} restricted & INT \\\hline
\multicolumn{3}{|p{\descWidth}|}{{\bf Description:}   {\em How many iterations to preallocate in extensible HDF5 datasets}} \\
\hline{\bf Range} & &  {\bf Default:} 200 \\\multicolumn{1}{|p{\maxVarWidth}|}{\centering 1:*} & \multicolumn{2}{p{\paraWidth}|}{Any integer} \\\hline
\end{tabular*}

\vspace{0.5cm}\noindent \begin{tabular*}{\tableWidth}{|c|l@{\extracolsep{\fill}}r|}
\hline
\multicolumn{1}{|p{\maxVarWidth}}{integration\_method} & {\bf Scope:} restricted & KEYWORD \\\hline
\multicolumn{3}{|p{\descWidth}|}{{\bf Description:}   {\em How to do surface integrals}} \\
\hline{\bf Range} & &  {\bf Default:} midpoint \\\multicolumn{1}{|p{\maxVarWidth}|}{\centering midpoint} & \multicolumn{2}{p{\paraWidth}|}{Midpoint rule (2nd order)} \\\multicolumn{1}{|p{\maxVarWidth}|}{\centering trapezoidal} & \multicolumn{2}{p{\paraWidth}|}{Trapezoidal rule (2nd order)} \\\multicolumn{1}{|p{\maxVarWidth}|}{\centering Simpson} & \multicolumn{2}{p{\paraWidth}|}{Simpson's rule (4th order) [requires even ntheta and nphi]} \\\multicolumn{1}{|p{\maxVarWidth}|}{\centering DriscollHealy} & \multicolumn{2}{p{\paraWidth}|}{Driscoll \& Healy (exponentially convergent) [requires even ntheta]} \\\hline
\end{tabular*}

\vspace{0.5cm}\noindent \begin{tabular*}{\tableWidth}{|c|l@{\extracolsep{\fill}}r|}
\hline
\multicolumn{1}{|p{\maxVarWidth}}{interpolator\_name} & {\bf Scope:} restricted & STRING \\\hline
\multicolumn{3}{|p{\descWidth}|}{{\bf Description:}   {\em Which interpolator should I use}} \\
\hline{\bf Range} & &  {\bf Default:} Hermite polynomial interpolation \\\multicolumn{1}{|p{\maxVarWidth}|}{\centering .+} & \multicolumn{2}{p{\paraWidth}|}{Any nonempty string} \\\hline
\end{tabular*}

\vspace{0.5cm}\noindent \begin{tabular*}{\tableWidth}{|c|l@{\extracolsep{\fill}}r|}
\hline
\multicolumn{1}{|p{\maxVarWidth}}{interpolator\_pars} & {\bf Scope:} restricted & STRING \\\hline
\multicolumn{3}{|p{\descWidth}|}{{\bf Description:}   {\em Parameters for the interpolator}} \\
\hline{\bf Range} & &  {\bf Default:} order=3 boundary\_off\_centering\_tolerance=\{0.0 0.0 0.0 0.0 0.0 0.0\} boundary\_extrapolation\_tolerance=\{0.0 0.0 0.0 0.0 0.0 0.0\} \\\multicolumn{1}{|p{\maxVarWidth}|}{\centering .*} & \multicolumn{2}{p{\paraWidth}|}{"Any string that Util\_TableSetFromStr 
ing() will take"} \\\hline
\end{tabular*}

\vspace{0.5cm}\noindent \begin{tabular*}{\tableWidth}{|c|l@{\extracolsep{\fill}}r|}
\hline
\multicolumn{1}{|p{\maxVarWidth}}{l\_max} & {\bf Scope:} restricted & INT \\\hline
\multicolumn{3}{|p{\descWidth}|}{{\bf Description:}   {\em The maximum l mode to extract}} \\
\hline{\bf Range} & &  {\bf Default:} 2 \\\multicolumn{1}{|p{\maxVarWidth}|}{\centering 0:*} & \multicolumn{2}{p{\paraWidth}|}{l {\textgreater}= 0} \\\hline
\end{tabular*}

\vspace{0.5cm}\noindent \begin{tabular*}{\tableWidth}{|c|l@{\extracolsep{\fill}}r|}
\hline
\multicolumn{1}{|p{\maxVarWidth}}{l\_min} & {\bf Scope:} restricted & INT \\\hline
\multicolumn{3}{|p{\descWidth}|}{{\bf Description:}   {\em all modes: above which l mode to calculate/ specific mode: which l mode to extract}} \\
\hline{\bf Range} & &  {\bf Default:} -1 \\\multicolumn{1}{|p{\maxVarWidth}|}{\centering -1:*} & \multicolumn{2}{p{\paraWidth}|}{Deprecated} \\\hline
\end{tabular*}

\vspace{0.5cm}\noindent \begin{tabular*}{\tableWidth}{|c|l@{\extracolsep{\fill}}r|}
\hline
\multicolumn{1}{|p{\maxVarWidth}}{l\_mode} & {\bf Scope:} restricted & INT \\\hline
\multicolumn{3}{|p{\descWidth}|}{{\bf Description:}   {\em The maximum l mode to extract}} \\
\hline{\bf Range} & &  {\bf Default:} -1 \\\multicolumn{1}{|p{\maxVarWidth}|}{\centering -1:*} & \multicolumn{2}{p{\paraWidth}|}{Deprecated} \\\hline
\end{tabular*}

\vspace{0.5cm}\noindent \begin{tabular*}{\tableWidth}{|c|l@{\extracolsep{\fill}}r|}
\hline
\multicolumn{1}{|p{\maxVarWidth}}{m\_mode} & {\bf Scope:} restricted & INT \\\hline
\multicolumn{3}{|p{\descWidth}|}{{\bf Description:}   {\em all modes: Up to which m mode to calculate/ specific mode: which m mode to extract }} \\
\hline{\bf Range} & &  {\bf Default:} -100 \\\multicolumn{1}{|p{\maxVarWidth}|}{\centering -100:*} & \multicolumn{2}{p{\paraWidth}|}{Deprecated} \\\hline
\end{tabular*}

\vspace{0.5cm}\noindent \begin{tabular*}{\tableWidth}{|c|l@{\extracolsep{\fill}}r|}
\hline
\multicolumn{1}{|p{\maxVarWidth}}{mode\_type} & {\bf Scope:} restricted & KEYWORD \\\hline
\multicolumn{3}{|p{\descWidth}|}{{\bf Description:}   {\em Which type of mode extraction do we have}} \\
\hline{\bf Range} & &  {\bf Default:} deprecated \\\multicolumn{1}{|p{\maxVarWidth}|}{\centering all modes} & \multicolumn{2}{p{\paraWidth}|}{Extract all modes up to (l\_mode, m\_mode).} \\\multicolumn{1}{|p{\maxVarWidth}|}{\centering specific mode} & \multicolumn{2}{p{\paraWidth}|}{Select one specific (l\_mode, m\_mode) mode} \\\multicolumn{1}{|p{\maxVarWidth}|}{\centering deprecated} & \multicolumn{2}{p{\paraWidth}|}{Deprecated} \\\hline
\end{tabular*}

\vspace{0.5cm}\noindent \begin{tabular*}{\tableWidth}{|c|l@{\extracolsep{\fill}}r|}
\hline
\multicolumn{1}{|p{\maxVarWidth}}{nphi} & {\bf Scope:} restricted & INT \\\hline
\multicolumn{3}{|p{\descWidth}|}{{\bf Description:}   {\em The number of points in the phi direction, minus one. (E.g., if this is set to 100, then 101 points will be chosen.)}} \\
\hline{\bf Range} & &  {\bf Default:} 100 \\\multicolumn{1}{|p{\maxVarWidth}|}{\centering 1:*} & \multicolumn{2}{p{\paraWidth}|}{} \\\hline
\end{tabular*}

\vspace{0.5cm}\noindent \begin{tabular*}{\tableWidth}{|c|l@{\extracolsep{\fill}}r|}
\hline
\multicolumn{1}{|p{\maxVarWidth}}{nradii} & {\bf Scope:} restricted & INT \\\hline
\multicolumn{3}{|p{\descWidth}|}{{\bf Description:}   {\em How many extraction radii?}} \\
\hline{\bf Range} & &  {\bf Default:} 1 \\\multicolumn{1}{|p{\maxVarWidth}|}{\centering 0:100} & \multicolumn{2}{p{\paraWidth}|}{} \\\hline
\end{tabular*}

\vspace{0.5cm}\noindent \begin{tabular*}{\tableWidth}{|c|l@{\extracolsep{\fill}}r|}
\hline
\multicolumn{1}{|p{\maxVarWidth}}{ntheta} & {\bf Scope:} restricted & INT \\\hline
\multicolumn{3}{|p{\descWidth}|}{{\bf Description:}   {\em The number of points in the theta direction, minus one. (E.g., if this is set to 50, then 51 points will be chosen.)}} \\
\hline{\bf Range} & &  {\bf Default:} 50 \\\multicolumn{1}{|p{\maxVarWidth}|}{\centering 0:*} & \multicolumn{2}{p{\paraWidth}|}{Positive please} \\\hline
\end{tabular*}

\vspace{0.5cm}\noindent \begin{tabular*}{\tableWidth}{|c|l@{\extracolsep{\fill}}r|}
\hline
\multicolumn{1}{|p{\maxVarWidth}}{out\_1d\_every} & {\bf Scope:} restricted & INT \\\hline
\multicolumn{3}{|p{\descWidth}|}{{\bf Description:}   {\em How often to output 1d data}} \\
\hline{\bf Range} & &  {\bf Default:} (none) \\\multicolumn{1}{|p{\maxVarWidth}|}{\centering } & \multicolumn{2}{p{\paraWidth}|}{no output} \\\multicolumn{1}{|p{\maxVarWidth}|}{\centering 1:*} & \multicolumn{2}{p{\paraWidth}|}{output every to many iterations} \\\hline
\end{tabular*}

\vspace{0.5cm}\noindent \begin{tabular*}{\tableWidth}{|c|l@{\extracolsep{\fill}}r|}
\hline
\multicolumn{1}{|p{\maxVarWidth}}{out\_dir} & {\bf Scope:} restricted & STRING \\\hline
\multicolumn{3}{|p{\descWidth}|}{{\bf Description:}   {\em Output directory for Extract's files, overrides IO::out\_dir}} \\
\hline{\bf Range} & &  {\bf Default:} (none) \\\multicolumn{1}{|p{\maxVarWidth}|}{\centering .+} & \multicolumn{2}{p{\paraWidth}|}{A valid directory name} \\\multicolumn{1}{|p{\maxVarWidth}|}{\centering \^\$} & \multicolumn{2}{p{\paraWidth}|}{An empty string to choose the default from IO::out\_dir} \\\hline
\end{tabular*}

\vspace{0.5cm}\noindent \begin{tabular*}{\tableWidth}{|c|l@{\extracolsep{\fill}}r|}
\hline
\multicolumn{1}{|p{\maxVarWidth}}{out\_every} & {\bf Scope:} restricted & INT \\\hline
\multicolumn{3}{|p{\descWidth}|}{{\bf Description:}   {\em How often to output}} \\
\hline{\bf Range} & &  {\bf Default:} 1 \\\multicolumn{1}{|p{\maxVarWidth}|}{\centering } & \multicolumn{2}{p{\paraWidth}|}{no output} \\\multicolumn{1}{|p{\maxVarWidth}|}{\centering 1:*} & \multicolumn{2}{p{\paraWidth}|}{output every to many iterations} \\\hline
\end{tabular*}

\vspace{0.5cm}\noindent \begin{tabular*}{\tableWidth}{|c|l@{\extracolsep{\fill}}r|}
\hline
\multicolumn{1}{|p{\maxVarWidth}}{output\_ascii} & {\bf Scope:} restricted & BOOLEAN \\\hline
\multicolumn{3}{|p{\descWidth}|}{{\bf Description:}   {\em Output a simple ASCII file for each mode at each radius}} \\
\hline & & {\bf Default:} yes \\\hline
\end{tabular*}

\vspace{0.5cm}\noindent \begin{tabular*}{\tableWidth}{|c|l@{\extracolsep{\fill}}r|}
\hline
\multicolumn{1}{|p{\maxVarWidth}}{output\_hdf5} & {\bf Scope:} restricted & BOOLEAN \\\hline
\multicolumn{3}{|p{\descWidth}|}{{\bf Description:}   {\em Output an HDF5 file for each variable containing one dataset per mode at each radius}} \\
\hline & & {\bf Default:} no \\\hline
\end{tabular*}

\vspace{0.5cm}\noindent \begin{tabular*}{\tableWidth}{|c|l@{\extracolsep{\fill}}r|}
\hline
\multicolumn{1}{|p{\maxVarWidth}}{radius} & {\bf Scope:} restricted & REAL \\\hline
\multicolumn{3}{|p{\descWidth}|}{{\bf Description:}   {\em The radii for extraction}} \\
\hline{\bf Range} & &  {\bf Default:} 0.0 \\\multicolumn{1}{|p{\maxVarWidth}|}{\centering 0.0:*} & \multicolumn{2}{p{\paraWidth}|}{Please keep it in the grid} \\\hline
\end{tabular*}

\vspace{0.5cm}\noindent \begin{tabular*}{\tableWidth}{|c|l@{\extracolsep{\fill}}r|}
\hline
\multicolumn{1}{|p{\maxVarWidth}}{test\_l} & {\bf Scope:} restricted & INT \\\hline
\multicolumn{3}{|p{\descWidth}|}{{\bf Description:}   {\em which mode to put into the test variables}} \\
\hline{\bf Range} & &  {\bf Default:} 2 \\\multicolumn{1}{|p{\maxVarWidth}|}{\centering *} & \multicolumn{2}{p{\paraWidth}|}{Any integer} \\\hline
\end{tabular*}

\vspace{0.5cm}\noindent \begin{tabular*}{\tableWidth}{|c|l@{\extracolsep{\fill}}r|}
\hline
\multicolumn{1}{|p{\maxVarWidth}}{test\_m} & {\bf Scope:} restricted & INT \\\hline
\multicolumn{3}{|p{\descWidth}|}{{\bf Description:}   {\em which mode to put into the test variables}} \\
\hline{\bf Range} & &  {\bf Default:} 2 \\\multicolumn{1}{|p{\maxVarWidth}|}{\centering *} & \multicolumn{2}{p{\paraWidth}|}{Any integer} \\\hline
\end{tabular*}

\vspace{0.5cm}\noindent \begin{tabular*}{\tableWidth}{|c|l@{\extracolsep{\fill}}r|}
\hline
\multicolumn{1}{|p{\maxVarWidth}}{test\_mode\_proportional\_to\_r} & {\bf Scope:} restricted & BOOLEAN \\\hline
\multicolumn{3}{|p{\descWidth}|}{{\bf Description:}   {\em whether the test spherical harmonic coefficient is proportional to the radial coordinate}} \\
\hline & & {\bf Default:} no \\\hline
\end{tabular*}

\vspace{0.5cm}\noindent \begin{tabular*}{\tableWidth}{|c|l@{\extracolsep{\fill}}r|}
\hline
\multicolumn{1}{|p{\maxVarWidth}}{test\_sw} & {\bf Scope:} restricted & INT \\\hline
\multicolumn{3}{|p{\descWidth}|}{{\bf Description:}   {\em which spin weight to put into the test variables}} \\
\hline{\bf Range} & &  {\bf Default:} -2 \\\multicolumn{1}{|p{\maxVarWidth}|}{\centering *} & \multicolumn{2}{p{\paraWidth}|}{Any integer} \\\hline
\end{tabular*}

\vspace{0.5cm}\noindent \begin{tabular*}{\tableWidth}{|c|l@{\extracolsep{\fill}}r|}
\hline
\multicolumn{1}{|p{\maxVarWidth}}{variables} & {\bf Scope:} restricted & STRING \\\hline
\multicolumn{3}{|p{\descWidth}|}{{\bf Description:}   {\em What variables to decompose}} \\
\hline{\bf Range} & &  {\bf Default:} (none) \\\multicolumn{1}{|p{\maxVarWidth}|}{\centering .*} & \multicolumn{2}{p{\paraWidth}|}{A list of variables} \\\hline
\end{tabular*}

\vspace{0.5cm}\noindent \begin{tabular*}{\tableWidth}{|c|l@{\extracolsep{\fill}}r|}
\hline
\multicolumn{1}{|p{\maxVarWidth}}{verbose} & {\bf Scope:} restricted & BOOLEAN \\\hline
\multicolumn{3}{|p{\descWidth}|}{{\bf Description:}   {\em Output detailed information about what is happening}} \\
\hline & & {\bf Default:} no \\\hline
\end{tabular*}

\vspace{0.5cm}\noindent \begin{tabular*}{\tableWidth}{|c|l@{\extracolsep{\fill}}r|}
\hline
\multicolumn{1}{|p{\maxVarWidth}}{io\_out\_dir} & {\bf Scope:} shared from IO & STRING \\\hline
\end{tabular*}

\vspace{0.5cm}\parskip = 10pt 

\section{Interfaces} 


\parskip = 0pt

\vspace{3mm} \subsection*{General}

\noindent {\bf Implements}: 

multipole
\vspace{2mm}

\noindent {\bf Inherits}: 

grid
\vspace{2mm}
\subsection*{Grid Variables}
\vspace{5mm}\subsubsection{PUBLIC GROUPS}

\vspace{5mm}

\begin{tabular*}{150mm}{|c|c@{\extracolsep{\fill}}|rl|} \hline 
~ {\bf Group Names} ~ & ~ {\bf Variable Names} ~  &{\bf Details} ~ & ~\\ 
\hline 
harmonics &  & compact & 0 \\ 
 & harmonic\_re & description & Spherical harmonics \\ 
 & harmonic\_im & dimensions & 3 \\ 
 &  & distribution & DEFAULT \\ 
 &  & group type & GF \\ 
 &  & timelevels & 1 \\ 
 &  & variable type & REAL \\ 
\hline 
test\_integration\_convergence\_orders &  & compact & 0 \\ 
 & test\_midpoint\_convergence\_order & description & Test integration convergence orders \\ 
 & test\_trapezoidal\_convergence\_order & dimensions & 0 \\ 
 & test\_simpson\_convergence\_order & distribution & CONSTANT \\ 
 &  & group type & SCALAR \\ 
 &  & timelevels & 1 \\ 
 &  & variable type & REAL \\ 
\hline 
test\_integration\_results &  & compact & 0 \\ 
 & test\_midpoint\_result\_low & description & Test integration results \\ 
 & test\_midpoint\_result\_high & dimensions & 0 \\ 
 & test\_trapezoidal\_result\_low & distribution & CONSTANT \\ 
 & test\_trapezoidal\_result\_high & group type & SCALAR \\ 
 & test\_simpson\_result\_low & timelevels & 1 \\ 
 & test\_simpson\_result\_high & variable type & REAL \\ 
\hline 
test\_integration\_symmetries &  & compact & 0 \\ 
 & test\_midpoint\_pi\_symmetry & description & Test integration symmetries \\ 
 & test\_trapezoidal\_pi\_symmetry & dimensions & 0 \\ 
 & test\_simpson\_pi\_symmetry & distribution & CONSTANT \\ 
 & test\_driscollhealy\_pi\_symmetry & group type & SCALAR \\ 
 &  & timelevels & 1 \\ 
 &  & variable type & REAL \\ 
\hline 
test\_orthonormality & test\_orthonormality & compact & 0 \\ 
 &  & description & Test orthonormality of spherical harmonics \\ 
 &  & dimensions & 2 \\ 
 &  & distribution & CONSTANT \\ 
 &  & group type & ARRAY \\ 
 &  & size & 1 \\ 
& ~ & size & (10*10)*(10*10+1)/2 \\ 
 &  & timelevels & 1 \\ 
 &  & variable type & REAL \\ 
\hline 
\end{tabular*} 



\vspace{5mm}\parskip = 10pt 

\section{Schedule} 


\parskip = 0pt


\noindent This section lists all the variables which are assigned storage by thorn EinsteinAnalysis/Multipole.  Storage can either last for the duration of the run ({\bf Always} means that if this thorn is activated storage will be assigned, {\bf Conditional} means that if this thorn is activated storage will be assigned for the duration of the run if some condition is met), or can be turned on for the duration of a schedule function.


\subsection*{Storage}

\hspace{5mm}

 \begin{tabular*}{160mm}{ll} 
~& {\bf Conditional:} \\ 
~ &  harmonics[1]\\ 
~ &  test\_integration\_convergence\_orders\\ 
~ &  test\_integration\_results\\ 
~ &  test\_integration\_symmetries\\ 
~ &  test\_orthonormality\\ 
~ & ~\\ 
\end{tabular*} 


\subsection*{Scheduled Functions}
\vspace{5mm}

\noindent {\bf CCTK\_ANALYSIS}   (conditional) 

\hspace{5mm} multipole\_calc 

\hspace{5mm}{\it calculate multipoles } 


\hspace{5mm}

 \begin{tabular*}{160mm}{cll} 
~ & After:  & calc\_np \\ 
~& ~ &psikadelia\\ 
~& ~ &accelerator\_copyback\\ 
~ & Language:  & c \\ 
~ & Options:  & global \\ 
~ & Type:  & function \\ 
\end{tabular*} 


\vspace{5mm}

\noindent {\bf CCTK\_INITIAL}   (conditional) 

\hspace{5mm} multipole\_setharmonic 

\hspace{5mm}{\it populate grid functions with spherical harmonics } 


\hspace{5mm}

 \begin{tabular*}{160mm}{cll} 
~ & Language:  & c \\ 
~ & Reads:  & grid::coordinates(everywhere) \\ 
~ & Type:  & function \\ 
~ & Writes:  & multipole::harmonics(interior) \\ 
\end{tabular*} 


\vspace{5mm}

\noindent {\bf CCTK\_INITIAL}   (conditional) 

\hspace{5mm} multipole\_testorthonormality 

\hspace{5mm}{\it loop over modes and integrate them to check orthonormality } 


\hspace{5mm}

 \begin{tabular*}{160mm}{cll} 
~ & Language:  & c \\ 
~ & Type:  & function \\ 
\end{tabular*} 


\vspace{5mm}

\noindent {\bf CCTK\_PARAMCHECK}   (conditional) 

\hspace{5mm} multipole\_paramcheck 

\hspace{5mm}{\it check multipole parameters } 


\hspace{5mm}

 \begin{tabular*}{160mm}{cll} 
~ & Language:  & c \\ 
~ & Options:  & global \\ 
~ & Type:  & function \\ 
\end{tabular*} 


\vspace{5mm}

\noindent {\bf CCTK\_PARAMCHECK}   (conditional) 

\hspace{5mm} multipole\_testintegrationconvergence 

\hspace{5mm}{\it test convergence of integration } 


\hspace{5mm}

 \begin{tabular*}{160mm}{cll} 
~ & Language:  & c \\ 
~ & Type:  & function \\ 
\end{tabular*} 


\vspace{5mm}

\noindent {\bf CCTK\_PARAMCHECK}   (conditional) 

\hspace{5mm} multipole\_testintegrationsymmetry 

\hspace{5mm}{\it test symmetry of integration } 


\hspace{5mm}

 \begin{tabular*}{160mm}{cll} 
~ & Language:  & c \\ 
~ & Type:  & function \\ 
\end{tabular*} 



\vspace{5mm}\parskip = 10pt 
\end{document}
